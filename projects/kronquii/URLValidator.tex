% template created by: Russell Haering. arr. Joseph Crop
\documentclass[12pt,letterpaper]{article}
\usepackage{enumerate}
\usepackage{anysize}
\marginsize{2cm}{2cm}{1cm}{1cm}

\begin{document}

\begin{titlepage}
    \vspace*{4cm}
    \begin{flushright}
    {\huge
        CS 362 Software Engineering II\\[1cm]
        Final Project\\[1cm]
    }
    \bigskip
    Ian Kronquist
    \vfill
    \end{flushright}
\end{titlepage}

\bigskip
\section{Finding Bugs}
\begin{enumerate}
    \item Manual Testing. For manual testing I tried a variety of URIs which used different protocols than http and https. I copied and built upon preexisting code in the URLValidatorTest file. I did not find any bugs initially.
    \begin{verbatim}
        UrlValidator urlValidator = new UrlValidator();
        assertTrue("osuosl.org should validate",
                validator.isValid("http://osuosl.org/test/index.html?foo=bar"));
        assertFalse("Git URIs are not URLs",
                validator.isValid("git://osuosl.org/test/index.html?foo=bar"));
        assertFalse("Gopher URIs are not URLS",
                validator.isValid("gopher://osuosl.org/test/index.html?foo=bar"));
        assertFalse("IRC URIs are not URLs",
                validator.isValid("irc://chat.freenode.net"));
        assertFalse("osuosl.org should still validate",
                validator.isValid("irc://chat.freenode.net"));
        assertTrue("Freenode's URL is valid",
                validator.isValid("http://chat.freenode.net"));
        assertFalse("Crazy nonsense should not be valid",
                validator.isValid("http://chat--.freenode.net"));
        assertFalse("URLs with emoji should not be valid",
                validator.isValid("http://🚀.♥︎"));
        assertTrue("test.invalidtld should validate because tld's aren't validated",
                validator.isValid("http://test.invalidtld/"));
    \end{verbatim}
    \item Input partitioning. By following Agan's first rule of debugging and inspecting and understanding the source code I determined that the tests partition the input into several categories: URL scheme, authority, port, path, and query. The default schemes are "http", "https", and "ftp". The schemes, authorities, ports, paths, and queries are all validated by different regexes.
    \item I wrote a simple unit test, contained in the file\\
    \texttt{/projects/kronquii/URLValidator/URLValidator/test/MyURLTest.java}, \\
    which combined various valid and invalid parts of input together. I was able to generate 237 test failures pretty quickly. By following Agan's rules of debugging and inspecting the URLs which failed, I was able to determine that the bug likely lay in the query section. Something also seemed off about the port section as well. I decided to start by focusing on the queries. After I determined there was a bug in the \texttt{isValidQuery} method, I realized that the \texttt{PORT\_REGEX} was incorrect.
    \item Teamwork: I reached out to my team, Jesse Poor and James Guerra, several times via their onid email accounts and they never replied. I tried to be accommodating with my schedule and offered them a broad swath of my time to meet. I hope I am not marked down because they were not communicative. It is frustrating to do an entire group project yourself. I have not heard from James at all, and I have not heard from Jesse since the 8th of November. Here are a couple of the most recent emails I sent them:\\
    \begin{enumerate}

    \item Dec 1, 2015\\
    Software Engineering II Final Project\\
    Hello,\\
    Are you guys interested in working together on the final project? Do you want to meet sometime to work on it? I am free most mornings before noon and most evenings after 6:00pm. I am in Corvallis on Pacific Time.\\

    Sincerely,\\
    Ian Kronquist\\
    \item Dec 5, 2015\\
    Re: Software Engineering II Final Project\\
    Hey guys,\\
    The final project expects us to work together. Are you interested in working together at all? If so, please let me know. I am available pretty much all day Sunday and in the evening this Saturday.\\

    Sincerely,\\
    Ian Kronquist\\
    \end{enumerate}
\end{enumerate}

\bigskip

\section{Bug Reports}
\begin{enumerate}
    \item URL query parameters are incorrectly marked as invalid or valid\\
    \textbf{Type:} Bug\\
    \textbf{Date Created:} 12/05/2015\\
    \textbf{Status:} Closed\\
    \textbf{Priority:} Normal\\
    \textbf{Components:} URLValidator\\
    \textbf{Reporter:} Ian Kronquist\\

    \textbf{Found by:} Automatic testing using input partitioning. I investigated the failure by setting breakpoints in the \texttt{isValidQuery} method and stepping through it.\\
    \textbf{Cause of failure:} The return value of the \texttt{isValidQuery} method was inverted.\\
    \textbf{Minimal test case:}
    \begin{verbatim}
        UrlValidator urlValidator = new UrlValidator();
        assertTrue("Should be a valid URL to search google",
                validator.isValid("https://www.google.com/search?q=test"));
        assertTrue("Should not be a valid URL",
                validator.isValid("https://www.google.com/search/q=noquestionmark"));
    \end{verbatim}


    \item URLs with ports greater than 999 are marked as invalid\\
    \textbf{Type:} Bug\\
    \textbf{Date Created:} 12/05/2015\\
    \textbf{Status:} Closed\\
    \textbf{Priority:} Normal\\
    \textbf{Components:} URLValidator\\
    \textbf{Reporter:} Ian Kronquist\\

    \textbf{Found by:} Automatic testing using input partitioning. I investigated the failure by setting breakpoints in the \texttt{isValidAuthority} method (which matches on the \texttt{PORT\_PATTERN} regex) and stepping through it. To confirm the exact nature of the bug I tested a variety of port values manually as well.\\
    \textbf{Cause of failure:} The value of the \texttt{PORT\_PATTERN} regex is incorrect.\\
    \textbf{Minimal test case:}
    \begin{verbatim}
        UrlValidator urlValidator = new UrlValidator();
        assertTrue("2^16-1 should be a valid port",
                validator.isValid("https://example.com:65535"));

        assertTrue("1000 should be a valid port",
                validator.isValid("https://example.com:1000"));
    \end{verbatim}



    \item Once \#3 is fixed, 4 digit ports greater than 65535 are marked as correct\\
    \textbf{Type:} Bug\\
    \textbf{Date Created:} 12/05/2015\\
    \textbf{Status:} Closed\\
    \textbf{Priority:} Normal\\
    \textbf{Components:} URLValidator\\
    \textbf{Reporter:} Ian Kronquist\\

    \textbf{Found by:} Playing a hunch after fixing \#3.\\
    \textbf{Cause of failure:} The value of the port is only validated with the \texttt{PORT\_PATTERN} regex. The port value is not checked.\\
    \textbf{Minimal test case:}
    \begin{verbatim}
        UrlValidator urlValidator = new UrlValidator();
        assertTrue("2^16-1 should be a valid port",
                validator.isValid("https://example.com:65536"));

        assertTrue("1000 should be a valid port",
                validator.isValid("https://example.com:99999"));
    \end{verbatim}
\end{enumerate}
\end{document}
